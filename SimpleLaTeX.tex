% ----------------------------------------------------------------- %
% This code was downloaded from simple-latex repository of GitHub user 'chanrt %
% Fork me at github.com/chanrt %
% ----------------------------------------------------------------- %

\documentclass[12pt]{article}

\usepackage{enumerate}
\usepackage{graphicx}
\usepackage{fullpage}
\usepackage{amssymb}
\usepackage{amsmath}
\usepackage{xcolor}
\usepackage[most]{tcolorbox}
\usepackage[colorlinks = true,
            linkcolor = blue,
            urlcolor  = blue,
            citecolor = blue,
            anchorcolor = blue]{hyperref}
            
% SYMBOLS %
\newcommand{\esc}{\textbackslash}
\newcommand{\imp}{\ensuremath{\Rightarrow}}
\newcommand{\degree}{\ensuremath{^{\circ}}}
\newcommand{\proportional}{\ensuremath{ \ \alpha \ }}
\newcommand{\complex}{\ensuremath{\mathbb{C} \ }}
\newcommand{\reals}{\ensuremath{\mathbb{R} \ }}
\newcommand{\rationals}{\ensuremath{\mathbb{Q} \ }}
\newcommand{\integers}{\ensuremath{\mathbb{Z} \ }}
\newcommand{\naturals}{\ensuremath{\mathbb{N} \ }}

% NUMBERS %
\newcommand{\num}[2]{\ensuremath{#1 \times 10^{#2}}}
\newcommand{\ten}[1]{\ensuremath{10^{#1}}}
\newcommand{\report}[3]{\ensuremath{(#1 \pm #2) \times 10^{#3}}}

% CONTAINERS %
\newcommand{\abs}[1]{\ensuremath{\left| #1 \right|}}
\newcommand{\floor}[1]{\ensuremath{\lfloor #1 \rfloor}}
\newcommand{\ceil}[1]{\ensuremath{\lceil #1 \rceil}}
\newcommand{\braces}[1]{\ensuremath{\left\lbrace #1 \right\rbrace}}
\newcommand{\angles}[1]{\ensuremath{\left\langle #1 \right\rangle}}
\newcommand{\bigbrac}[1]{\ensuremath{\left( #1 \right)}}

% FRACTIONS %
\newcommand{\reci}[1]{\ensuremath{\displaystyle\frac{1}{#1}}}
\newcommand{\bigfrac}[2]{\ensuremath{\bigbrac{\displaystyle\frac{#1}{#2}}}}

% PRESENTATION %
\newcommand{\super}[1]{\ensuremath{^{#1}}}
\newcommand{\sub}[1]{\ensuremath{_{#1}}}
\newcommand{\st}{\ensuremath{^{st}}}
\newcommand{\nd}{\ensuremath{^{nd}}}
\newcommand{\rd}{\ensuremath{^{rd}}}
\newcommand{\nth}{\ensuremath{^{th}}}
\newcommand{\atconstant}[2]{\ensuremath{\bigbrac{#1}_{#2}}}
\newcommand{\seq}[2]{\ensuremath{#1_1, #1_2, #1_3, \dots #1_{#2}}}
\newcommand{\numeq}[1]{\ensuremath{ \ \dots \ (#1)}}

% ITERATION %
\newcommand{\summation}[3]{\ensuremath{\sum\limits_{#2}^{#3} #1}}
\newcommand{\product}[3]{\ensuremath{\prod\limits_{#2}^{#3} #1}}
\newcommand{\union}[3]{\ensuremath{\cup_{#2}^{#3} #1}}
\newcommand{\intersection}[3]{\ensuremath{\cap_{#2}^{#3} #1}}

% COMBINATORICS %
\newcommand{\perm}[2]{\ensuremath{ ^{#1}P _{#2}}}
\newcommand{\comb}[2]{\ensuremath{{#1 \choose #2}}}

% DIFFERENTIAL CALCULUS %
\newcommand{\slope}[2]{\ensuremath{\displaystyle\frac{\Delta #1}{\Delta #2}}}
\newcommand{\limitof}[3]{\ensuremath{\lim\limits_{#2 \rightarrow #3}} #1}
\newcommand{\der}[2]{\ensuremath{\displaystyle\frac{d #1}{d #2}}}
\newcommand{\nder}[3]{\ensuremath{\displaystyle\frac{d^{#3} #1}{d #2^{#3}}}}
\newcommand{\partialder}[2]{\ensuremath{\displaystyle\frac{\partial #1}{\partial #2}}}
\newcommand{\partialderk}[3]{\ensuremath{\bigbrac{\displaystyle\frac{\partial #1}{\partial #2}}_{#3}}}
\newcommand{\mixedder}[3]{\ensuremath{\displaystyle\frac{\partial^2 #1}{\partial #2 \partial #3}}}
\newcommand{\mixedderk}[4]{\ensuremath{\bigbrac{\displaystyle\frac{\partial^2 #1}{\partial #2 \partial #3}}_{#4}}}
\newcommand{\npartialder}[3]{\ensuremath{\displaystyle\frac{\partial^{#3} #1}{\partial #2^{#3}}}}
\newcommand{\npartialderk}[4]{\ensuremath{\bigbrac{\displaystyle\frac{\partial^{#3} #1}{\partial #2^{#3}}}_{#4}}}

% INTEGRAL CALCULUS %
\newcommand{\indefint}[3]{\ensuremath{\int\limits_{#2}^{#3} #1}}
\newcommand{\defint}[3]{\ensuremath{\left|#1\right|_{#2}^{#3}}}
\newcommand{\indefintinf}[1]{\ensuremath{\int\limits_{-\infty}^{+\infty} #1}}
\newcommand{\defintinf}[1]{\ensuremath{\left|#1\right|_{-\infty}^{\infty}}}

% PHYSICS SYMBOLS %
\newcommand{\xdot}{\ensuremath{\dot{x}}}
\newcommand{\xddot}{\ensuremath{\ddot{x}}}
\newcommand{\pdot}{\ensuremath{\dot{p}}}
\newcommand{\pddot}{\ensuremath{\ddot{p}}}
\newcommand{\qdot}{\ensuremath{\dot{q}}}
\newcommand{\qddot}{\ensuremath{\ddot{q}}}
\newcommand{\hamil}{\ensuremath{\mathcal{H}}}
\newcommand{\lagr}{\ensuremath{\mathcal{L}}}

% ORBITALS %
\newcommand{\paired}{\ensuremath{\boxed{\uparrow \downarrow}}}
\newcommand{\unpaired}{\ensuremath{\boxed{\uparrow \ }}}

% FORMATTING %
\pagecolor{white}
\color{black}
\newcommand{\drawline}{\vspace{5mm}\hrule}
\newcommand{\nextpage}{\vspace{5mm}\hrule\newpage}
\newcommand{\point}[1]{\item\textbf{#1:}}

% GRAPHICS %
\newcommand{\image}[2]{\begin{center}\includegraphics[scale=#2]{#1}\end{center}}  
\newcommand{\setboxcolor}[1]{\tcbset{
    frame code={}
    center title,
    left=0pt,
    right=0pt,
    top=0pt,
    bottom=0pt,
    colback=#1,
    colframe=black,
    width=\dimexpr\textwidth\relax,
    enlarge left by=0mm,
    boxsep=5pt,
    arc=0pt,outer arc=0pt,    
    }}
\newcommand{\insidebox}[2]{\setboxcolor{#1}\begin{tcolorbox}#2\end{tcolorbox}}
\newcommand{\highlight}[1]{\insidebox{light-gray}{#1}}

% COLORS %
\definecolor{white}{RGB}{255 255 255}
\definecolor{black}{RGB}{0 0 0}
\definecolor{violet}{RGB}{127 0 255}
\definecolor{indigo}{RGB}{75 0 120}
\definecolor{blue}{RGB}{0 0 255}
\definecolor{green}{RGB}{0 255 0}
\definecolor{yellow}{RGB}{255 255 0}
\definecolor{orange}{RGB}{255 69 0}
\definecolor{red}{RGB}{255 0 0}
\definecolor{purple}{RGB}{128 0 128}
\definecolor{magenta}{RGB}{255 0 255}
\definecolor{cyan}{RGB}{0 255 255}
\definecolor{light-gray}{RGB}{215 215 215}
\definecolor{custom-color}{RGB}{0 0 0}

% DECLARATIONS %
\newcommand{\nocopy}[2]{\vspace{3mm}\section*{Declaration} I, #1, hereby declare that I have compiled this file without obtaining help from anyone. \image{#2}{0.3} \drawline}
\newcommand{\thankyou}{\begin{center}\textbf{Thank You}\end{center}}

\begin{document}

\highlight{\textbf{Simple LaTeX}}
Fork me at \href{https://www.github.com/chanrt}{GitHub}.

\section{Symbols}

\begin{itemize}

\point{Escape character} \esc esc prints \esc
\point{Implies} \esc imp prints \imp
\point{Degrees} \esc degree prints \degree
\point{Proportional} \esc proportional prints \proportional with appropriate gap
\point{Number sets} \esc complex, \esc reals, \esc rationals, \esc integers, \esc naturals prints \complex, \reals, \rationals, \integers, and \naturals respectively, with appropriate gap

\end{itemize}
\drawline

\section{Numbers}

\begin{itemize}

\point{General number} \esc num \braces{num1} \braces{num2} prints \num{num1}{num2}
\point{Power of ten} \esc ten \braces{power} prints \ten{power}
\point{Reporting a figure} \esc report \braces{num1} \braces{error} \braces{num2} prints \report{num1}{error}{num2}

\end{itemize}
\drawline

\section{Containers}

\begin{itemize}

\point{Absolute value} \esc abs \braces{num} prints \abs{num}
\point{Floor} \esc floor \braces{num} prints \floor{num}
\point{Ceiling} \esc ceil \braces{num} prints \ceil{num}
\point{Brace brackets} \esc braces \braces{num} prints \braces{num}
\point{Angular brackets} \esc angles \braces{num} prints \angles{num}
\point{Big brackets} \esc bigbrac \braces{num} inside \$\$ \$\$ prints brackets of appropriate size  

\end{itemize}
\drawline

\section{Fractions}

\begin{itemize}

\point{Reciprocal} \esc reci \braces{num} prints \reci{num}
\point{Big fraction} \esc bigfrac \braces{num1} \braces{num2} \ prints a fraction inside brackets of appropriate size

\end{itemize}
\drawline

\section{Presentation}

\begin{itemize}

\point{Superscript} num1 \esc super \braces{num2} prints num1\super{num2}
\point{Subscript} num1 \esc sub \braces{num2} prints num1\sub{num2}
\point{Ordinal numbers} \esc st, \esc nd \esc rd and \esc nth print \st, \nd, \rd, \nth respectively
\point{Expression evaluated at a constant value} \esc atconstant \braces{exp} \braces{constant} prints \atconstant{exp}{constant} with brackets of appropriate size
\point{Sequence} \esc seq \braces{x} \braces{n} prints \seq{x}{n}
\point{Numbering an equation} \esc numeq \braces{2} prints \numeq{2} 

\end{itemize}
\drawline

\section{Iterations}

\begin{itemize}

\point{Summation} \esc summation \braces{x} \braces{x=1} \braces{n} prints \summation{x}{x=1}{n}
\point{Product} \esc product \braces{x} \braces{x=1} \braces{n} prints \product{x}{x=1}{n}
\point{Union} \esc union \braces{A_x} \braces{x=1} \braces{n} prints \union{A_x}{x=1}{n}
\point{Intersection} \esc intersection \braces{A_x} \braces{x=1} \braces{n} prints \intersection{A_x}{x=1}{n}

\end{itemize}
\drawline

\section{Combinatorics}

\begin{itemize}

\point{Permutation} \esc perm \braces{n} \braces{r} prints \perm{n}{r}
\point{Combination} \esc comb \braces{n} \braces{r} prints \comb{n}{r}

\end{itemize}
\drawline

\section{Differential Calculus}

\begin{itemize}

\point{Slope} \esc slope \braces{x} \braces{t} prints \slope{x}{t}

\point{Limits} \esc limitof \braces{f(x)} \braces{x} \braces{0} prints \limitof{f(x)}{x}{0}

\point{Derivative} \esc der \braces{x} \braces{t} prints \der{x}{t}

\point{n\nth Derivative} \esc nder \braces{x} \braces{t} \braces{n} prints \nder{x}{t}{n}

\point{Partial Derivative} \esc partialder \braces{x} \braces{t} prints \partialder{x}{t}

\point{n\nth Partial Derivative} \esc npartialder \braces{x} \braces{t} \braces{n} prints \npartialder{x}{t}{n}

\point{Mixed Derivative} \esc mixedder \braces{x} \braces{y} \braces{t} prints \mixedder{x}{y}{t}

\point{Evaluation at constant} Adding k to each command and an extra \braces{k}, evaluates partial derivatives at k. For example, \esc npartialderk \braces{x} \braces{t} \braces{3} \braces{k} prints \npartialderk{x}{t}{2}{k}

\end{itemize}
\drawline

\section{Integral Calculus}

\begin{itemize}

\point{Indefinite Integral} \esc indefint \braces{f(x) dx} \braces{a} \braces{b} prints \indefint{f(x) dx}{a}{b}

\point{Definite Integral} \esc defint \braces{f(x)} \braces{a} \braces{b} prints \defint{f(x)}{a}{b}

\point{Indefinite Integral at infinity} \esc indefintinf \braces{f(x) dx} prints \indefintinf{f(x) dx}

\point{Definite Integral at infinity} \esc defintinf \braces{f(x)} prints \defintinf{f(x)}

\end{itemize}
\nextpage

\end{document}